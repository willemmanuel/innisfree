% This is "sig-alternate.tex" V2.0 May 2012
% This file should be compiled with V2.5 of "sig-alternate.cls" May 2012
%
% This example file demonstrates the use of the 'sig-alternate.cls'
% V2.5 LaTeX2e document class file. It is for those submitting
% articles to ACM Conference Proceedings WHO DO NOT WISH TO
% STRICTLY ADHERE TO THE SIGS (PUBS-BOARD-ENDORSED) STYLE.
% The 'sig-alternate.cls' file will produce a similar-looking,
% albeit, 'tighter' paper resulting in, invariably, fewer pages.
%
% ----------------------------------------------------------------------------------------------------------------
% This .tex file (and associated .cls V2.5) produces:
%       1) The Permission Statement
%       2) The Conference (location) Info information
%       3) The Copyright Line with ACM data
%       4) NO page numbers
%
% as against the acm_proc_article-sp.cls file which
% DOES NOT produce 1) thru' 3) above.
%
% Using 'sig-alternate.cls' you have control, however, from within
% the source .tex file, over both the CopyrightYear
% (defaulted to 200X) and the ACM Copyright Data
% (defaulted to X-XXXXX-XX-X/XX/XX).
% e.g.
% \CopyrightYear{2007} will cause 2007 to appear in the copyright line.
% \crdata{0-12345-67-8/90/12} will cause 0-12345-67-8/90/12 to appear in the copyright line.
%
% ---------------------------------------------------------------------------------------------------------------
% This .tex source is an example which *does* use
% the .bib file (from which the .bbl file % is produced).
% REMEMBER HOWEVER: After having produced the .bbl file,
% and prior to final submission, you *NEED* to 'insert'
% your .bbl file into your source .tex file so as to provide
% ONE 'self-contained' source file.
%
% ================= IF YOU HAVE QUESTIONS =======================
% Questions regarding the SIGS styles, SIGS policies and
% procedures, Conferences etc. should be sent to
% Adrienne Griscti (griscti@acm.org)
%
% Technical questions _only_ to
% Gerald Murray (murray@hq.acm.org)
% ===============================================================
%
% For tracking purposes - this is V2.0 - May 2012

\documentclass{sig-alternate}

\begin{document}
%
% --- Author Metadata here ---
\conferenceinfo{WOODSTOCK}{'97 El Paso, Texas USA}
%\CopyrightYear{2007} % Allows default copyright year (20XX) to be over-ridden - IF NEED BE.
%\crdata{0-12345-67-8/90/01}  % Allows default copyright data (0-89791-88-6/97/05) to be over-ridden - IF NEED BE.
% --- End of Author Metadata ---

\title{Alternate {\ttlit ACM} SIG Proceedings Paper in LaTeX
Format\titlenote{(Produces the permission block, and
copyright information). For use with
SIG-ALTERNATE.CLS. Supported by ACM.}}
\subtitle{[Extended Abstract]
\titlenote{A full version of this paper is available as
\textit{Author's Guide to Preparing ACM SIG Proceedings Using
\LaTeX$2_\epsilon$\ and BibTeX} at
\texttt{www.acm.org/eaddress.htm}}}
%
% You need the command \numberofauthors to handle the 'placement
% and alignment' of the authors beneath the title.
%
% For aesthetic reasons, we recommend 'three authors at a time'
% i.e. three 'name/affiliation blocks' be placed beneath the title.
%
% NOTE: You are NOT restricted in how many 'rows' of
% "name/affiliations" may appear. We just ask that you restrict
% the number of 'columns' to three.
%
% Because of the available 'opening page real-estate'
% we ask you to refrain from putting more than six authors
% (two rows with three columns) beneath the article title.
% More than six makes the first-page appear very cluttered indeed.
%
% Use the \alignauthor commands to handle the names
% and affiliations for an 'aesthetic maximum' of six authors.
% Add names, affiliations, addresses for
% the seventh etc. author(s) as the argument for the
% \additionalauthors command.
% These 'additional authors' will be output/set for you
% without further effort on your part as the last section in
% the body of your article BEFORE References or any Appendices.

\numberofauthors{6} %  in this sample file, there are a *total*
% of EIGHT authors. SIX appear on the 'first-page' (for formatting
% reasons) and the remaining two appear in the \additionalauthors section.
%
\author{
% You can go ahead and credit any number of authors here,
% e.g. one 'row of three' or two rows (consisting of one row of three
% and a second row of one, two or three).
%
% The command \alignauthor (no curly braces needed) should
% precede each author name, affiliation/snail-mail address and
% e-mail address. Additionally, tag each line of
% affiliation/address with \affaddr, and tag the
% e-mail address with \email.
%
% 1st. author
\alignauthor
Bethany Connor\\
       \affaddr{University of Virginia}\\
       \email{bac5rc@virginia.edu}
% 2nd. author
\alignauthor
Robert Emerson\\
       \affaddr{University of Virginia}\\
       \email{roe2pj@virginia.edu}
% 3rd. author
\alignauthor
Will Emmanuel\\
       \affaddr{University of Virginia}\\
       \email{wre9fz@virginia.edu}
\and  % use '\and' if you need 'another row' of author names
% 4th. author
\alignauthor
Xavier Palathingal\\
       \affaddr{University of Virginia}\\
       \email{xvp2he@virginia.edu}
% 5th. author
\alignauthor
Domenic Puzio\\
       \affaddr{University of Virginia}\\
       \email{dvp5qd@virginia.edu}
% 6th. author
\alignauthor
Huiqing Zhang\\
       \affaddr{University of Virginia}\\
       \email{hz9cx@virginia.edu}
}
% There's nothing stopping you putting the seventh, eighth, etc.
% author on the opening page (as the 'third row') but we ask,
% for aesthetic reasons that you place these 'additional authors'
% in the \additional authors block, viz.

% Just remember to make sure that the TOTAL number of authors
% is the number that will appear on the first page PLUS the
% number that will appear in the \additionalauthors section.

\maketitle
\begin{abstract}
This paper provides a sample of a \LaTeX\ document which conforms,
somewhat loosely, to the formatting guidelines for
ACM SIG Proceedings. It is an {\em alternate} style which produces
a {\em tighter-looking} paper and was designed in response to
concerns expressed, by authors, over page-budgets.
It complements the document \textit{Author's (Alternate) Guide to
Preparing ACM SIG Proceedings Using \LaTeX$2_\epsilon$\ and Bib\TeX}.
This source file has been written with the intention of being
compiled under \LaTeX$2_\epsilon$\ and BibTeX.

The developers have tried to include every imaginable sort
of ``bells and whistles", such as a subtitle, footnotes on
title, subtitle and authors, as well as in the text, and
every optional component (e.g. Acknowledgments, Additional
Authors, Appendices), not to mention examples of
equations, theorems, tables and figures.

To make best use of this sample document, run it through \LaTeX\
and BibTeX, and compare this source code with the printed
output produced by the dvi file. A compiled PDF version
is available on the web page to help you with the
`look and feel'.
\end{abstract}

% A category with the (minimum) three required fields
\category{H.4}{Information Systems Applications}{Miscellaneous}
%A category including the fourth, optional field follows...
\category{D.2.8}{Software Engineering}{Metrics}[complexity measures, performance measures]

\terms{Theory}

\keywords{ACM proceedings, \LaTeX, text tagging}

\section{Introduction}
Innisfree Village is a non-profit organization located in Crozet, Virginia dedicated to providing a lifesharing community for adults with disabilities (Innisfree Village). In this community, about 40 adults, known as co-workers, live and work alongside 20 long-term volunteer caregivers in houses of 4 co-workers and 2 caregivers. In addition to these full-time volunteers, there are around a dozen part-time volunteers and a dozen staff members providing more specialized knowledge and care. Throughout the day, residents participate in a variety of activities to contribute to the community, including cooking, gardening, woodworking, and weaving. Volunteers, who have committed to serving at the village for a year and a half, spend much of their time working together with their co-workers and helping meet their needs. Meanwhile, staff members are responsible for much of the administrative and maintenance work necessary to keep the community thriving.

As a full-time residential community, Innisfree Village is also responsible for scheduling medical appointments and ensuring co-workers can make these appointments. This is one of the main tasks for the staff, requiring one staff position, the medical coordinator, to be fully devoted to co-worker medical care, while also involving many others. Given that the 40 co-workers of Innisfree Village have varying disabilities and medical needs, ensuring all the necessary appointments have been made and can be attended is vital. Our work for them will focus on overhauling this scheduling system. The current system used for this is primarily pen and paper. When an appointment is made, the medical coordinator writes it on a large calendar in the main office, before recording it in Excel for reporting purposes. When a co-worker has an appointment, that co-worker’s volunteer caregiver is responsible for taking them to the appointment.

One of the major issues with this system is its lack of responsiveness. Not only is it exceedingly slow, as it requires several steps to simply record an appointment, but caregivers are rarely able to make follow-up appointments in the doctor’s office, as it also requires several phone calls between the medical coordinator and the doctor to schedule an appointment. The current system also requires appointments to be entered twice, increasing the secretarial load required, and increasing the possibilities of errors. It also makes it hard to generate reports for specific houses or residents, as Excel doesn’t provide the same features as a full database system. Finally, it is not simple for the medical coordinator to remind caregivers about upcoming appointments. Since all appointments are recorded on the calendar, the medical coordinator writes paper reminders for each caregiver, and has to hope the caregiver will check their mailbox in time.


\section{Related Work}
There are several applications that were discussed with Innisfree Village as possible solutions.  The first one discussed was mainstream online scheduling options such as Google Calendar {does this need a citation?}.  While our system was eventually modeled on the view of Google Calendars, this kind of calendar was too generic for what our customers wanted.  It does not allow for the management of residents, doctors, and volunteers, there are few user levels, and they are difficult to filter.  In addition, these calendars are made to be managed by one person and viewed by many, which did not fit in with the needs of this system. Finally, these applications do not allow data to be exported in a report friendly format, something our customer required.
	
Another scheduling option included desktop clients, such as Outlook, as these also have a developed calendar.  However, programs like this don’t fit with the way volunteers would use the site, which is mostly on their cell phones and would not support car sign-out features.  As with Google Calendar, there is no way to manage user levels.

\section{System Design}

\subsection{Users}

\subsubsection{Users Model}
	The user model lists all of the attributes for system users. This includes the user’s name, email, house, email preferences, and admin status, as well as the auto-generated Ruby on Rails Id, created at and updated at fields. Since we used the Devise gem {devise} for user authentication, it generated and automatically fills fields for the user's password, password reset, and login information. The user name, email and password are all required fields when a new user is created. The house field is not required, but is always filled out at account creation time. As there is a one-to-many relationship between houses and users, where each house has many users and each user has one house, when the house field of a user is changed, it is validated against the houses in the database to ensure that house actually exists.

\subsubsection{Viewing, Creating, Editing, and Deleting Users}
	The users are displayed in two places. The first is a user directory, accessible to all users, where a user can look up the house, email address, or phone number of another user. Users can also edit their own profile from this page. The second user display an administrative interface that only administrator users can access. Here, administrators can mark other users as administrators or as medical coordinators, turn user emails on and off, edit user profiles and delete users.

\subsection{Appointments}

\subsubsection{Appointments Model}
The appointments model contains attributes for the resident, the doctor, the date, the time, the volunteer assigned to the appointment, the appointment type, a notes section, whether the appointment has been canceled and the Ruby on Rails generated id, created at, and updated at fields.  In order to create an appointment, the resident id (from the Resident table), the doctor id (from the Doctor table), the appointment type, the date, and the time must be specified.  The notes field user id of the volunteer assigned to the appointment defaults to NULL and the canceled field defaults to false.  Id, created at, and updated at are all purely managed by Ruby on Rails. 

In addition, the customer requested that we have a way of tracking updates to an appointment.  For this, we used the Rails gem PaperTrail. {papertrail}  This created a separate table named versions that kept track of each of these events for any model we marked as being tracked.  Namely, this table contains a copy of the object as well as the type of the object.

\subsubsection{Viewing, Creating, Editing, and Deleting Appointments}
The appointments are displayed on the landing page of the site in both a list and calendar format.  In the calendar view, displayed using the fullcalendar Rails gem {fullcalendar}, we chose to only show the number of appointments on that given day in an effort to reduce clutter.  However, each day has the option to click on it to see the full list of appointment on that day.  The list view on the landing page is a paginated list of the next ten appointments.

Creating appointments can be done by both admin and volunteers.  There is a button above the appointments display to create a new appointment.  This brings up a form that has fields for all fields of the appointments model except the Rails managed fields and the canceled field.  All fields are drop-down menus except for the date, time, and notes fields.  The items in these drop-down menus are pulled straight from other tables of the database, and, as the only table volunteers can create entries into is the appointments table, are managed by the admin of the site.

As the project progressed, we realized that outright deleting an appointment did not constitute the same action as canceling one.  Therefore, we added an option to cancel an appointment.  These appointments then still appeared anywhere they would've before cancellation but with a strike-through.  This allowed for better communication between the various employees of the Village and to demonstrate that an appointment had indeed been made.  Volunteers and admin can both cancel an appointment (and uncancel it if the cancellation was done in error) but only admin can outright delete an appointment.      

\subsection{Doctors}

\subsubsection{Doctors Model}
The doctors model contains attributes for each doctor’s name, address, phone number and doctor type. All doctors must at least have values for their names. 

\subsubsection{Creating, Viewing, Editing and Deleting Doctors}
Only admins can create, edit and delete doctors. In order to create a new doctor, the user must enter the name of that doctor. Users can view the information about each doctor by clicking on their names. Users can also click on each doctor’s address and they will be redirected to location of that address on google map. 

\subsection{Reports}
On Reports page, admins can generate reports for the complete record of residents' appointments. They have the option to filter each report's content by houses, residents, doctors, appointment type and date range. Two types of formats are available for these reports, CSV and PDF. Users can choose one of the format before they generate the report. CSV files can be easily imported into Excel and other data management software for further editing. PDF files have logo of Innisfree Village in the heading and they are well-formatted with clear headings and tables. Users can easily read them and print them out for filing purposes. 

There is a second type of reports that users can generate in Houses page, Home page, Doctors page and Users page. Users can generate a CSV file that includes all the context that are currently displayed on that particular page. For example, if the user chooses to generate this type of CSV report for doctors, that user will go to Doctors page and click on "Download as CSV" and download a CSV file that contains all the information including every doctor's name, address, doctor type and phone number. 

\subsection{E-mail Notifications}
At the beginning of development, e-mail notifications were limited to simply e-mailing volunteers the morning of appointments that were assigned to them.  However, as the project progressed, we added more notifications including weekly digests, reminders to schedule appointments, and the option to remind volunteers of a specific appointment.  In addition, who was sent these notifications was changed; instead of only being to the volunteer assigned, it was sent to the volunteers of the house of the applicable resident.

In order to send e-mails, we used ActionMailer, which comes with Rails.  For each type of e-mail we sent, we wrote a new function in NotificationMailer, which was our extension of ActionMailer.  Then, in our User model, we defined who these functions were sent to.  For example, we have a method called send\_weekly\_digest.   This iterates through the users, selects the ones that are medical coordinators (only medical coordinators get the weekly digest), then calls the weekly\_digest method in NotificationMailer.  A schedule defined in config/schedule.rb defines how often to call certain functions; for example in the case of the weekly digest, it is sent at 6am on every Sunday.  Other types of notifications are triggered by an action on the site and are therefore not scheduled in schedule.rb. 

%ACKNOWLEDGMENTS are optional
\section{Acknowledgments}
This section is optional; it is a location for you
to acknowledge grants, funding, editing assistance and
what have you.  In the present case, for example, the
authors would like to thank Gerald Murray of ACM for
his help in codifying this \textit{Author's Guide}
and the \textbf{.cls} and \textbf{.tex} files that it describes.

%
% The following two commands are all you need in the
% initial runs of your .tex file to
% produce the bibliography for the citations in your paper.
\bibliographystyle{abbrv}
\bibliography{sigproc}  % sigproc.bib is the name of the Bibliography in this case
% You must have a proper ".bib" file
%  and remember to run:
% latex bibtex latex latex
% to resolve all references
%
% ACM needs 'a single self-contained file'!
%
%APPENDICES are optional
%\balancecolumns
\appendix
%Appendix A
\section{Headings in Appendices}
The rules about hierarchical headings discussed above for
the body of the article are different in the appendices.
In the \textbf{appendix} environment, the command
\textbf{section} is used to
indicate the start of each Appendix, with alphabetic order
designation (i.e. the first is A, the second B, etc.) and
a title (if you include one).  So, if you need
hierarchical structure
\textit{within} an Appendix, start with \textbf{subsection} as the
highest level. Here is an outline of the body of this
document in Appendix-appropriate form:
\subsection{Introduction}
\subsection{The Body of the Paper}
\subsubsection{Type Changes and  Special Characters}
\subsubsection{Math Equations}
\paragraph{Inline (In-text) Equations}
\paragraph{Display Equations}
\subsubsection{Citations}
\subsubsection{Tables}
\subsubsection{Figures}
\subsubsection{Theorem-like Constructs}
\subsubsection*{A Caveat for the \TeX\ Expert}
\subsection{Conclusions}
\subsection{Acknowledgments}
\subsection{References}
Generated by bibtex from your ~.bib file.  Run latex,
then bibtex, then latex twice (to resolve references)
to create the ~.bbl file.  Insert that ~.bbl file into
the .tex source file and comment out
the command \texttt{{\char'134}thebibliography}.
% This next section command marks the start of
% Appendix B, and does not continue the present hierarchy
\section{More Help for the Hardy}
The sig-alternate.cls file itself is chock-full of succinct
and helpful comments.  If you consider yourself a moderately
experienced to expert user of \LaTeX, you may find reading
it useful but please remember not to change it.
%\balancecolumns % GM June 2007
% That's all folks!
\end{document}
